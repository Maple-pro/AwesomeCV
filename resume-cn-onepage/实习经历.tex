%-------------------------------------------------------------------------------
%	SECTION TITLE
%-------------------------------------------------------------------------------
\cvsection{实习经历}


%-------------------------------------------------------------------------------
%	CONTENT
%-------------------------------------------------------------------------------
\begin{cventries}

%---------------------------------------------------------
  \cventry
    {Java 后端开发实习生} % 岗位
    {蚂蚁集团 \quad 收银产品技术组} % 公司 部门
    {2024.06 - 至今} % 时间
    {杭州} % 地点
    {
      \begin{cvitems} % Description(s) of tasks/responsibilities
        \item {开发收银台相关的业务功能:使用不同类型的业务渠道支付订单,为用户处理优惠,用户安全付款并展示给用户付款结果;}
        \item {组合支付支持输入金额:为支付宝的组合支付功能提供手动分配金额能力,推演了分配金额过程中用户的状态变更并设计了服务端协议,完成了收银台服务端的代码开发;}
        \item {收银台渠道 mock 标准化:梳理历史 mock 渠道,将收银台首页渠道 mock 的方式抽象出一个接入点,使得渠道 mock 做到标准化,便于接入。}
      \end{cvitems}
    }

%---------------------------------------------------------
  \cventry
    {前端开发实习生} % 岗位
    {蚂蚁集团 \quad 体验技术部} % 公司 部门
    {2021.06 - 2021.09} % 时间
    {杭州} % 地点
    {
      \begin{cvitems} % Description(s) of tasks/responsibilities
        \item {负责开发余额宝前端的相关业务功能,使用 Node.js 开发 BFF 中间层,使用 React.js 框架开发前端界面;}
        \item {参与了余额宝「工资日」营销活动的开发,负责其中前端部分的需求分析和红包组件的开发,提升了余额宝的日活量。}
      \end{cvitems}
    }

%---------------------------------------------------------
  % \cventry
  %   {C\#\ .NET 开发{\enskip\cdotp\enskip}需求分析} % 岗位
  %   {武汉大学 \quad 交叉信息研究中心实验室} % 公司 部门
  %   {2019.12 - 2021.03} % 时间
  %   {武汉} % 地点
  %   {
  %     \begin{cvitems} % Description(s) of tasks/responsibilities
  %       \item {参与 AutoCAD 自动布线项目的开发,负责与用户的需求沟通和需求分析;}
  %       \item {使用 C\# .NET 负责 AutoCAD 二次开发相关的工作,实现自动布线算法与 AutoCAD 之间的交互层。}
  %     \end{cvitems}
  %   }

%---------------------------------------------------------
\end{cventries}
