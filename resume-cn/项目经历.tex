%-------------------------------------------------------------------------------
%	SECTION TITLE
%-------------------------------------------------------------------------------
\cvsection{项目经历}


%-------------------------------------------------------------------------------
%	CONTENT
%-------------------------------------------------------------------------------
\begin{cventries}

%---------------------------------------------------------
  \cventry
    {Go 语言微服务开发} % Role
    {微服务抖音后端 GuGoTik} % 项目名称
    {2023.08 - 2023.09} % 时间
    {} %
    {
      \begin{cvitems} % Description(s) of tasks/responsibilities
        \item {项目描述:使用 Golang 实现了一个微服务分布式极简版抖音后端,具有视频智能推送和热度视频引流、ChatGPT 集成等特色;}
        \item {技术架构:Golang, Gin, gRPC, Redis, RabbitMQ, OpenTelemetry, Kubernetes, Envoy, Consul, Pyroscope, Open API, K6;}
        \item {个人职责:}
          \begin{itemize}
            % \item {基于 Gin 和 gRPC 框架实现视频的评论 RPC 接口和 HTTP 接口,使用基于 Redis 的令牌桶算法实现评论阈值的检测和限制,避免恶意攻击导致服务不可用;}
            \item {评论风控:通过 RabbitMQ 对评论做异步的风控处理,将文字内容辅以特定的 Prompt 输入到 Open API 中获取该评论的情感倾向,并根据情感倾向做风控处理;}
            % \item {基于 Gin 和 gRPC 框架实现视频的上传 RPC 接口和 HTTP 接口,使用 RabbitMQ 对上传的视频做异步的视频流处理,提升了服务的可用性;}
            \item {视频流处理:使用 ffmpeg 提取视频封面和音频文件,通过 Open API 从音频文件中提取文字内容,并将文字内容辅以特定的 Prompt 输入到 Open API 中生成视频内容总结和视频关键词,用于后续的视频推荐算法中;}
            \item {AGI 接入:接入了 Open API 用于评论风控和视频关键词提取,使用延迟队列和死信队列解决了 Open API 访问阈值限制的问题;}
            \item {多级缓存:使用 Redis + 业务本地缓存实现了多级缓存架构,提高了系统的可用性,解决了 Redis 中的热 key 问题;}
            % \item {基于 Redis 和布隆过滤器实现了登录过滤,减少数据库查询和防止恶意攻击;}
            \item {审计日志:使用基于 RabbitMQ 的消息总线采集用户信息用于视频推荐并生成审计表;}
            \item {Feed 流:使用推拉结合的方式实现了视频的 Feed 流,并使用多级缓存和消息队列增强 Feed 流的可用性。}
          \end{itemize}
        \item {项目成果:实现了一个微服务分布式极简版抖音后端,具有基本的视频功能、互动功能和社交功能;所有接口均全链路无错误,压力测试中90\%的请求时长为 23ms;最终获得了第六届字节跳动后端青训营的第一名。}
      \end{cvitems}
    }

%---------------------------------------------------------
  \cventry
    {程序分析{\enskip\cdotp\enskip}测试生成算法开发{\enskip\cdotp\enskip}Android Studio 插件开发} % Role
    {面向 Android 的 Kotlin 程序单元测试用例自动生成工具 KotSuite(OPPO 校企合作项目)} % 项目名称
    {2022.09 - 2023.12} % 时间
    {} %
    {
      \begin{cvitems} % Description(s) of tasks/responsibilities
        \item {项目描述:目前 OPPO 面临较大的软件交付压力,手动开发单元测试用例的效率低,因此我们开发了一个针对 Android 项目的测试用例自动生成工具 KotSuite,使用「演化算法」和「测试用例重用算法」对 Android 项目中的 Java/Kotlin 代码自动生成单元测试用例,包含一个命令行工具和 Android Studio 插件;}
        \item {技术架构:Java、Soot(静态分析)、JaCoCo(覆盖信息分析)、Fernflower(反编译工具)、Android Studio 插件开发;}
        \item {个人职责:}
          \begin{itemize}
            \item {基于「演化算法」和「测试用例重用算法」实现了测试用例生成算法,是第一个针对 Kotlin 和 Android 测试用例自动生成工具;}
            \item {开发了一个 Android Studio 插件以优化用户体验;}
          \end{itemize}
        \item {项目成果:在 OPPO 内部的实际项目中生成的测试用例的行覆盖率平均值达到了 80\% - 90\%,已在公司内部落地使用。}
      \end{cvitems}
    }

%---------------------------------------------------------
  \cventry
    {Java 语言微服务开发{\enskip\cdotp\enskip}后端开发{\enskip\cdotp\enskip}前端开发} % Role
    {牛备备工程设备托管租赁平台} % 项目名称
    {2022.03 - 2023.06} % 时间
    {} %
    {
      \begin{cvitems}
        \item {项目描述:为了解决工程设备租赁双方信息不对称、难以即时交易的问题,我们开发了一个用于工程设备托管租赁的平台牛备备,支持用户在平台上传设备、下订单租赁设备,以及工作人员审核等功能,接入了支付宝支付接口,后端使用 SpringBoot 微服务架构,前端包含一个用于工程设备托管租赁的微信小程序,以及一个用于后台管理的 Web 界面;}
        \item {技术架构:Java、SpringBoot、MySQL、Kafka、ElasticSearch、uni-app、Vue.js;}
        \item {个人职责:}
          \begin{itemize}
            \item {登录鉴权:使用 JWT 方案实现用户鉴权,基于 Redis 实现了布隆过滤器用于登录过滤,减少数据库查询,防止恶意登录攻击;}
            \item {订单模块:基于支付宝 SDK 封装了支付相关接口,基于 MySQL 设计订单的存储结构;}
            \item {审计日志:使用基于 Kafka 的消息总线采集设备、订单等信息,并使用 EFK 架构将审计日志存入 ElasticSearch,便于搜索;}
            \item {分销机制:参与设计三级分销机制,并负责设计分销信息的存储结构,并结合支付宝 SDK 和 Kafka 自动发送分销奖励;}
            \item {邀请有礼:开发「邀请有礼」营销活动,包括邀请码的生成和识别,结合支付宝 SDK 自动发送邀请奖励等;}
            \item {前端:使用 uni-app 参与部分小程序前端界面的开发。}
          \end{itemize}
        \item {项目成果:项目落地使用并稳定运行,用户数量达 300 余人,设备数量达 1000 余台。}
      \end{cvitems}
    }

%---------------------------------------------------------
  % \cventry
  %   {Golang 开发{\enskip\cdotp\enskip}分布式课程实验} % Role
  %   {基于 Raft 共识算法的 KV-store 框架} % 项目名称
  %   {2023.12 - 2024.02} % 时间
  %   {} %
  %   {
  %     \begin{cvitems}
  %       \item {项目描述:MIT6.824 分布式课程的课程实验,使用 Golang 实现 Raft 协议,并基于 Raft 协议实现一个 multi-raft(Raft 集群) 的 KV 存储服务。}
  %     \end{cvitems}
  %   }

%---------------------------------------------------------
  % \cventry
  %   {智能合约开发{\enskip\cdotp\enskip}前端开发{\enskip\cdotp\enskip}项目经理} % Role
  %   {基于以太坊区块链的点赞众筹平台 TokenGo} % 项目名称
  %   {2020.10 - 2021.03} % 时间
  %   {} %
  %   {
  %     \begin{cvitems}
  %       \item {项目描述:一个基于以太坊区块链的DApp,用户可以在平台上发布自己的创意项目,并能通过点赞(会消耗虚拟货币)为其他用户的创意项目众筹,当项目落地并有收益后,用户可以向点赞过这个项目的人发放虚拟货币,以达到良性循环;} 
  %       \item {技术架构:Truffle、Solidity、Vue.js;}
  %       \item {个人职责:}
  %         \begin{itemize}
  %           \item {使用 Truffle 框架开发了该平台,其中使用 Solidity 语言开发以太坊智能合约,使用 Vue.js 开发前端界面;}
  %           \item {参与了智能合约和前端界面的开发,并担任项目经理参与了技术选型和商业计划书的撰写。}
  %         \end{itemize}
  %     \end{cvitems}
  %   }

%---------------------------------------------------------
\end{cventries}
